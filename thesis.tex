\documentclass[a4paper,10pt]{memoir}
\usepackage[italian]{babel}
\usepackage{wrapfig}
\usepackage[pdftex]{graphicx}
\usepackage{graphviz}
\usepackage{amsmath}

\usepackage[chapter]{minted}
\usepackage{adjustbox}
\usepackage{hyperref}
\hypersetup{
  colorlinks   = true,    % Colours links instead of ugly boxes
  urlcolor     = black,    % Colour for external hyperlinks
  linkcolor    = black,    % Colour of internal links
  citecolor    = black      % Colour of citations
}

% import package
\usepackage{FrontespizioSapienza}

% declare info
\FSSTitolo{Titolo}
\FSSFacolta{Ingegneria dell'Informazione, Informatica e Statistica}
\FSSCorso{Informatica}

\FSSCandidato{Edoardo Ottavianelli}
\FSSMatricola{1756005}
\FSSRelatore{Emanuele Panizzi}
\FSSCorrelatore{}
\FSSAnnoAccademico{2019/2020}


\begin{document}

\frontmatter


% print title
\maketitle
\cleardoublepage

%\vspace*{10cm}
%\begin{flushright}
%\textsl{...}
%\end{flushright}
%\cleardoublepage

% ======================================= ABSTRACT ================================================
\begin{abstract}
	abstract
\end{abstract}
\cleardoublepage

\tableofcontents
\cleardoublepage

\mainmatter

\renewcommand\chapterheadstart{}
\renewcommand\printchaptername{}
\renewcommand\chapternamenum{}
\renewcommand\printchapternum{}
\renewcommand\afterchapternum{}
\renewcommand\printchaptertitle[1]{\chaptitlefont \thechapter. \space #1}


% ======================================= CHAPTER 1 ================================================
\chapter{Introduzione a SeismoCloud e obiettivi del progetto}

\section{I terremoti e la loro natura}

\section{Misurazione dei terremoti}

\section{Utilizzare l'informatica per rilevare terremoti}

\section{Piattaforma SeismoCloud}

\section{Bisogni della community SeismoCloud}


% ======================================= CHAPTER 2 ================================================
\chapter{Architettura di Node-RED adattata a SeismoCloud}

\section{Perché Node-RED}

\section{Difficoltà e limiti nell'uso di Node-RED}

\section{Pianificazione del progetto}

\section{Architettura adattata a SeismoCloud}

\section{Implementazione delle funzionalità tramite nodi}

\section{Ottimizzazione delle prestazioni}


% ======================================= CHAPTER 3 ================================================
\chapter{Sicurezza e protezione dei dati}

\section{Sicurezza nella piattaforma Node-RED}

\section{Ricerca ed analisi di vulnerabilità}

\section{Risoluzione dei problemi}


% ======================================= CHAPTER 4 ================================================
\chapter{Test, conclusioni e sviluppi futuri}

\section{Iterazioni dei Test}

\section{Conclusioni}

\section{Sviluppi futuri}


\refstepcounter{chapter}

% ======================================= BIBLIOGRAPHY ================================================
\begin{thebibliography}{9}

\bibitem{placeholder}
	\textbf{placeholder}.
	placeholder

\end{thebibliography}

\end{document}