\documentclass[a4paper,10pt]{memoir}
\usepackage[italian]{babel}
\usepackage{wrapfig}
\usepackage[pdftex]{graphicx}
\usepackage{graphviz}
\usepackage{amsmath}

\usepackage[chapter]{minted}
\usepackage{adjustbox}
\usepackage{hyperref}
\hypersetup{
  colorlinks   = true,    % Colours links instead of ugly boxes
  urlcolor     = black,    % Colour for external hyperlinks
  linkcolor    = black,    % Colour of internal links
  citecolor    = black      % Colour of citations
}

% import package
\usepackage{FrontespizioSapienza}

% declare info
\FSSTitolo{Titolo}
\FSSFacolta{Ingegneria dell'Informazione, Informatica e Statistica}
\FSSCorso{Informatica}

\FSSCandidato{Edoardo Ottavianelli}
\FSSMatricola{1756005}
\FSSRelatore{Emanuele Panizzi}
\FSSCorrelatore{}
\FSSAnnoAccademico{2019/2020}


\begin{document}

\frontmatter


% print title
\maketitle
\cleardoublepage

%\vspace*{10cm}
%\begin{flushright}
%\textsl{...}
%\end{flushright}
%\cleardoublepage

% ======================================= ABSTRACT ================================================
\begin{abstract}
	abstract
\end{abstract}
\cleardoublepage

\tableofcontents
\cleardoublepage

\mainmatter

\renewcommand\chapterheadstart{}
\renewcommand\printchaptername{}
\renewcommand\chapternamenum{}
\renewcommand\printchapternum{}
\renewcommand\afterchapternum{}
\renewcommand\printchaptertitle[1]{\chaptitlefont \thechapter. \space #1}


% ======================================= CHAPTER 1 ================================================
\chapter{Introduzione a SeismoCloud e obiettivi del progetto}

\section{I terremoti e la loro natura}

\begin{wrapfigure}[14]{r}{0.50\textwidth}
\caption{Composizione della Terra}
\label{fig:crostaterrestre}
\includegraphics[width=0.50\textwidth]{Chapter-1/crosta-terrestre.png}
\end{wrapfigure}

La terra è composta a strati, o meglio da involucri concentrici, come possiamo notare dalla figura 1.1, ed ognuno di essi ha diverse caratteristiche e particolarità.
Al centro della Terra c'è il \textbf{nucleo interno}, ossia un ammasso viscoso composto quasi esclusivamente da ferro avente un raggio di circa 1250 km. Si raggiungono temperature molto elevate, circa 5000-6000$^{\circ}$C.
A seguire abbiamo il \textbf{nucleo esterno}, principalmente composto per il 20\% da ferro e la restante parte nichel, raggiunge circa i 3000$^{\circ}$C. Comprendendo anche il nucleo interno, ha un raggio di circa 3500 km. \\

Continuando verso l'esterno, troviamo il \textbf{mantello terrestre}, che si divide in superiore ed inferiore.
È composto da diversi metalli ed è difficile stabilire la temperatura dato i moti convettivi del calore, ma si stima intorno ai 500$^{\circ}$C a confine con la crosta terrestre e 3000$^{\circ}$C a confine con il nucleo. \\
Infine abbiamo la \textbf{crosta terrestre}, che partendo dalla superficie, arriva fino a 70 km di profondità. \\
Insieme, il \textit{mantello superiore} e la \textit{crosta terrestre} formano la \textbf{litosfera}.

\section{Misurazione dei terremoti}

placeholder

\section{Utilizzare l'informatica per rilevare terremoti}

placeholder

\section{Piattaforma SeismoCloud}

placeholder

\section{Bisogni della community SeismoCloud}

placeholder

% ======================================= CHAPTER 2 ================================================
\chapter{Architettura di Node-RED adattata a SeismoCloud}

\section{Perché Node-RED}

\section{Difficoltà e limiti nell'uso di Node-RED}

\section{Pianificazione del progetto}

\section{Architettura adattata a SeismoCloud}

\section{Implementazione delle funzionalità tramite nodi}

\section{Ottimizzazione delle prestazioni}


% ======================================= CHAPTER 3 ================================================
\chapter{Sicurezza e protezione dei dati}

\section{Sicurezza nella piattaforma Node-RED}

\section{Ricerca ed analisi di vulnerabilità}

\section{Risoluzione dei problemi}


% ======================================= CHAPTER 4 ================================================
\chapter{Test, conclusioni e sviluppi futuri}

\section{Iterazioni dei Test}

\section{Conclusioni}

\section{Sviluppi futuri}


\refstepcounter{chapter}

% ======================================= BIBLIOGRAPHY ================================================
\begin{thebibliography}{9}

\bibitem{placeholder}
	\textbf{placeholder}.
	placeholder

\end{thebibliography}

\end{document}